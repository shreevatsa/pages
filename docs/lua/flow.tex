\documentclass{standalone}
\usepackage{tikz}
\usepackage{xcolor}
\usepackage{luacode}

\definecolor{c0}{HTML}{5A311D}
\definecolor{c1}{HTML}{E18B4E}
\definecolor{c2}{HTML}{4A1776}
\definecolor{c3}{HTML}{C966DA}
\definecolor{c4}{HTML}{04676C}
\definecolor{c5}{HTML}{0CE7E1}
\definecolor{c6}{HTML}{004692}
\definecolor{c7}{HTML}{0082FF}
\definecolor{c8}{HTML}{355128}
\definecolor{c9}{HTML}{DF1C24}

\begin{document}
\newcommand{\distance}{6}

\begin{tikzpicture}
  \foreach \digit in {0,1,...,9}{
    \foreach \position in {0,1,...,100}{
      % Want to allocate range [36d .. 36(d+1)) to digit, but use only say 90% of the range.
      \pgfmathsetlengthmacro{\angle}{\digit/10*360 + 0.90*\position/100/10*360}
      \coordinate (\digit-\position) at (\angle: \distance);
    }
  }
  \begin{luacode}
    dofile("pidigits.lua")
    function print_edges()
        local position = -1   -- Starting at -1 because pi_digits yields 0314159265...
        local previous = nil
        for digit in coroutine.wrap(pi_digits) do
            if position >= 1 then
                tex.print(string.format("\\path (%d-%d) edge [bend right, color=c%d] (%d-%d);",
                                        previous, position-1, previous, digit, position))
            end
            previous = digit
            position = position + 1
        end
    end
    print_edges()
  \end{luacode}
\end{tikzpicture}
\end{document}
