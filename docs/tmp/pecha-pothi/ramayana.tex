% The dimensions of the "main content" area and its offset on the page.
\hsize=5in \vsize=1.5in
\hoffset=-0.7in \voffset=-0.7in % Relative to the default of 1 inch.
\pdfpagewidth=6.5in \pdfpageheight=2.2in

% The material that should appear in the left and right bars.
% These should be redefined as the document progresses.
\def\booktitle{वा॰रा॰}
\def\chaptertitle{बा॰का॰}
\def\chapnumber{स॰ १}
\def\pagenumber{\number\pageno}

% The left-side "header"
\def\leftbar{%
  \vbox to \vsize{\hbox{\booktitle} \vss \hbox{\chaptertitle}}%
  \hskip 1pt%
  \vrule height \vsize depth 5pt%
  \hskip 1pt%
  \vrule height \vsize depth 5pt%
  }

% The right-side "header" (or is it "footer"?)
\def\rightbar{%
  \vrule height \vsize depth 5pt%
  \hskip 1pt%
  \vrule height \vsize depth 5pt%
  \hskip 1pt%
  \vbox to \vsize{\hbox{\chapnumber} \vss \hbox{\pagenumber}}%
  }

% The output routine: the page to \shipout, given main content in \box255.
\output={\shipout\hbox{%
  \leftbar%
  \hskip 10pt%
  \box255%
  \hskip 10pt%
  \rightbar%
  }\advancepageno}

% TECKit mapping=devanagarinumerals converts "1" to "१", etc.
\font\devaserif="Noto Serif Devanagari:mapping=devanagarinumerals"

\devaserif

तपःस्वाध्यायनिरतं तपस्वी वाग्विदां वरम् । नारदं परिपप्रच्छ वाल्मीकिर्मुनिपुंगवम् ॥ १ ॥
को न्वस्मिन् साम्प्रतं लोके गुणवान् कश्च वीर्यवान् । धर्मज्ञश्च कृतज्ञश्च सत्यवाक्यो दृढव्रतः ॥ २ ॥
चारित्रेण च को युक्तः सर्वभूतेषु को हितः । विद्वान् कः कः समर्थश्च कश्चैकप्रियदर्शनः ॥ ३ ॥
आत्मवान् को जितक्रोधो मतिमान् को ऽनसूयकः । कस्य बिभ्यति देवाश्च जातरोषस्य संयुगे ॥ 4 ॥
एतद् इच्छाम्यहं श्रोतुं परं कौतूहलं हि मे । महर्षे त्वं समर्थो ऽसि ज्ञातुम् एवंविधं नरम् ॥ 5 ॥
श्रुत्वा चैतत् त्रिलोकज्ञो वाल्मीकेर्नारदो वचः । श्रूयताम् इति चामन्त्र्य प्रहृष्टो वाक्यमब्रवीत् ॥ 6 ॥
बहवो दुर्लभाश्चैव ये त्वया कीर्तिता गुणाः । मुने वक्ष्याम्यहं बुद्ध्वा तैर्युक्तः श्रूयतां नरः ॥ 7 ॥
इक्ष्वाकुवंशप्रभवो रामो नाम जनैः श्रुतः । नियतात्मा महावीर्यो द्युतिमान् धृतिमान् वशी ॥ 8 ॥
बुद्धिमान् नीतिमान् वाग्मी श्रीमाञ् शत्रुनिबर्हणः । विपुलांसो महाबाहुः कम्बुग्रीवो महाहनुः ॥ 9 ॥
महोरस्को महेष्वासो गूढजत्रुररिंदमः । आजानुबाहुः सुशिराः सुललाटः सुविक्रमः ॥ 10 ॥
समः समविभक्ताङ्गः स्निग्धवर्णः प्रतापवान् । पीनवक्षा विशालाक्षो लक्ष्मीवाञ् शुभलक्षणः ॥ 11 ॥
धर्मज्ञः सत्यसंधश्च प्रजानां च हिते रतः । यशस्वी ज्ञानसंपन्नः शुचिर्वश्यः समाधिमान् ॥ 12 ॥
रक्षिता जीवलोकस्य धर्मस्य परिरक्षिता । वेदवेदाङ्गतत्त्वज्ञो धनुर्वेदे च निष्ठितः ॥ 13 ॥
सर्वशास्त्रार्थतत्त्वज्ञो स्मृतिमान् प्रतिभानवान् । सर्वलोकप्रियः साधुर् अदीनात्मा विचक्षणः ॥ 14 ॥
सर्वदाभिगतः सद्भिः समुद्र इव सिन्धुभिः । आर्यः सर्वसमश्चैव सदैकप्रियदर्शनः ॥ 15 ॥
स च सर्वगुणोपेतः कौसल्यानन्दवर्धनः । समुद्र इव गाम्भीर्ये धैर्येण हिमवान् इव ॥ 16 ॥
विष्णुना सदृशो वीर्ये सोमवत् प्रियदर्शनः । कालाग्निसदृशः क्रोधे क्षमया पृथिवीसमः ॥ 17 ॥
धनदेन समस् त्यागे सत्ये धर्म इवापरः । तम् एवंगुणसंपन्नं रामं सत्यपराक्रमम् ॥ 18 ॥
ज्येष्ठं श्रेष्ठगुणैर्युक्तं प्रियं दशरथः सुतम् । यौवराज्येन संयोक्तुम् ऐच्छत् प्रीत्या महीपतिः ॥ 19 ॥
तस्याभिषेकसंभारान् दृष्ट्वा भार्याथ कैकयी । पूर्वं दत्तवरा देवी वरम् एनम् अयाचत ॥ 20 ॥
विवासनं च रामस्य भरतस्याभिषेचनम् । स सत्यवचनाद् राजा धर्मपाशेन संयतः ॥ 21 ॥
विवासयाम् आस सुतं रामं दशरथः प्रियम् । स जगाम वनं वीरः प्रतिज्ञाम् अनुपालयन् ॥ 22 ॥
पितुर्वचननिर्देशात् कैकेय्याः प्रियकारणात् । तं व्रजन्तं प्रियो भ्राता लक्ष्मणो ऽनुजगाम ह ॥ 23 ॥
स्नेहाद् विनयसंपन्नः सुमित्रानन्दवर्धनः । सर्वलक्षणसंपन्ना नारीणाम् उत्तमा वधूः ॥ 24 ॥
सीताप्यनुगता रामं शशिनं रोहिणी यथा । पौरैरनुगतो दूरं पित्रा दशरथेन च ॥ 25 ॥

\medskip
तेन गत्वा पुरीं लङ्कां हत्वा रावणम् आहवे । अभ्यषिञ्चत् स लङ्कायां राक्षसेन्द्रं विभीषणम् ॥ 66 ॥
कर्मणा तेन महता त्रैलोक्यं सचराचरम् । सदेवर्षिगणं तुष्टं राघवस्य महात्मनः ॥ 67 ॥
तथा परमसंतुष्टैः पूजितः सर्वदैवतैः । कृतकृत्यस् तदा रामो विज्वरः प्रमुमोद ह ॥ 68  ॥
देवताभ्यो वरान् प्राप्य समुत्थाप्य च वानरान् । पुष्पकं तत् समारुह्य नन्दिग्रामं ययौ तदा ॥ 69 ॥
नन्दिग्रामे जटां हित्वा भ्रातृभिः सहितो ऽनघः । रामः सीताम् अनुप्राप्य राज्यं पुनरवाप्तवान् ॥ 70  ॥

\bye
